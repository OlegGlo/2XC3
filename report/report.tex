\documentclass[twocolumn, 10pt]{article}

\usepackage{amsmath}
\usepackage{graphicx}
\usepackage{float}
\graphicspath{ {./images/} }

\title{2XC3 - Lab 1 Report}
\author{Oleg Glotov\\ L03, 400174037\\ glotovo@mcmaster.ca \and Emma Willson\\ L02, 400309856\\ willsone@mcmaster.ca}


\begin{document}
\maketitle
\section{Git Setup}\label{sec:git}
The group began by setting up out github accounts and the repository. The visibility was set to private to comply with McMaster's code of conduct.

\begin{figure}[H]
\includegraphics[width=\linewidth]{img1}
\end{figure}

Both of us has access to the repository. We choose to work with github directly thorough our VSCode IDE since it fully integrates git commands and simplifies the workflow.

\section{Git commands}

We experimented with several git commands, including merge, reset and revert, which are shown below. 

Here we have merge demo

\begin{figure}[H]
\includegraphics[width=\linewidth]{merge}
\end{figure}

Here we have reset demo

\begin{figure}[H]
\includegraphics[width=\linewidth]{reset}
\end{figure}

Here we have revert demo

\begin{figure}[H]
\includegraphics[width=\linewidth]{revert}
\end{figure}

First, we experimented with merge by editing the same line in a text file and then pushing that change to the repository. 
In VS Code, we were shown the conflict and given the option to choose a version of the file, to merge the two, or reject both changes.\\
Then, we used reset to undo some changes made to this report file. Since reset moves the branch that HEAD points to, to a different commit, 
it's easy to undo uncommitted changes to the file. A downside to reset is that the undone changes are lost.\\ 
Finally, we used revert to undo some changes made to another file. Revert creates a new commit that reverses the changes made by the previous 
commit. This provides clear documentation of previous versions and allows us to easily return to those before the reverted one.


\section{Code.py}

Both group member's code is provided below. After consulting we decided to combine both of ours solution into the final product in the file code.py not provided here.

Oleg's version:

\footnotesize
\begin{verbatim}
def are_valid_groups(groups, studentNum):

    length = len(studentNum)
    count = 0
    
    for group in groups:
        count = 0
        for students in studentNum:
            if (group.count(students) == 0):
                break
            if (group.count(students) == 1):
                count += 1
        
        if count == length:
            return True
        
    return False

print(are_valid_groups(groups, studentNum))
\end{verbatim}
\normalsize

Emma's version:

\footnotesize
\begin{verbatim}
def are_valid_groups(nums, groups):
    valid = True
    for num in nums:
        valid = False
        for group in groups:
            if (num in group): 
                valid = True
                break
        if (not valid):
            return False
    return True

print(are_valid_groups(groups, studentNum))
\end{verbatim}
\normalsize

\section{Player vs Adversary}
Let the games begin
\end{document}

