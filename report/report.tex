\documentclass[twocolumn, 10pt]{article}

\usepackage{amsmath}
\usepackage{graphicx}
\usepackage{float}
\graphicspath{ {./images/} }

\title{2XC3 - Lab 1 Report}
\author{Oleg Glotov\\ L03, 400174037\\ glotovo@mcmaster.ca \and Emma Willson\\ L02, 400309856\\ willsone@mcmaster.ca}


\begin{document}
\maketitle
\section{Git Setup}\label{sec:git}
The group began by setting up out github accounts and the repository. The visibility was set to private to comply with McMaster's code of conduct.

\begin{figure}[H]
\includegraphics[width=\linewidth]{img1}
\end{figure}

Both of us has access to the repository. We choose to work with github directly thorough our VSCode IDE since it fully integrates git commands and simplifies the workflow.

\section{Git commands}

Here we fucked around more resulting in the shit you see below

We used revert to create a new commit that reverses the changes made by the previous commit. This preserves previous versions and allows us to return to those before the reverted one.  

and

The

other 

things

Merge demo

\begin{figure}[H]
\includegraphics[width=\linewidth]{merge}
\end{figure}

\section{Code.py}

Both group member's code is provided below. After consulting we decided to combine both of ours solution into the final product in the file code.py not provided here.

Oleg's version:

\footnotesize
\begin{verbatim}
def are_valid_groups(groups, studentNum):

    length = len(studentNum)
    count = 0
    
    for group in groups:
        count = 0
        for students in studentNum:
            if (group.count(students) == 0):
                break
            if (group.count(students) == 1):
                count += 1
        
        if count == length:
            return True
        
    return False

print(are_valid_groups(groups, studentNum))
\end{verbatim}
\normalsize

Emma's version:

\footnotesize
\begin{verbatim}
def are_valid_groups(nums, groups):
    valid = True
    for num in nums:
        valid = False
        for group in groups:
            if (num in group): 
                valid = True
                break
        if (not valid):
            return False
    return True

print(are_valid_groups(groups, studentNum))
\end{verbatim}
\normalsize

\section{Player vs Adversary}
Let the games begin
\end{document}

